% Lab3.tex
\documentclass[12pt,a4paper]{article}
\usepackage[utf8]{inputenc}
\usepackage[vietnamese]{babel}
\usepackage{amsmath,amssymb,amsthm}
\usepackage{enumitem}
\usepackage{geometry}
\geometry{margin=1in}

\title{LAB 3: Phân tích thuật toán (tiếp theo)\\\large }
\author{Nguyễn Tiến Dũng - 23001585}
\date{}

\begin{document}
\maketitle
\section*{Bài 1.1: Áp dụng Master Theorem}
\begin{enumerate}[label=\alph*.]
\item $T(n)=9T(n/3)+n$.

\textbf{Phân tích:} $a=9,\; b=3 \Rightarrow \log_b a=\log_3 9=2$. Ta có $f(n)=n = O(n^{2-\varepsilon})$ (ví dụ $\varepsilon=1$). Đây là \textbf{trường hợp 1}.

\textbf{Kết luận:} $T(n)=\Theta(n^{2})$.

\medskip
\item $T(n)=T(2n/3)+1$.

\textbf{Phân tích:} Viết dưới dạng $T(n)=a\,T(n/b)+f(n)$ ta có $a=1$ và $n/b = 2n/3 \Rightarrow b=\tfrac{3}{2}$. Do đó $\log_b a=\log_{3/2}1=0$. $f(n)=1=\Theta(n^{0})$, tương đương với $n^{\log_b a}$, nên là \textbf{trường hợp 2}.

\textbf{Kết luận:} $T(n)=\Theta(\log n)$.

\medskip
\item $T(n)=3T(n/4)+n\log n$.

\textbf{Phân tích:} $a=3,\; b=4 \Rightarrow \log_b a=\log_4 3\approx 0.79$. $f(n)=n\log n = n^{1}\log n$, rõ ràng lớn hơn $n^{\log_4 3}$ theo đa thức (vì mũ 1 > 0.79). Kiểm tra điều kiện regularity:
\[
a\,f(n/b)=3\cdot \frac{n}{4}\log\!\Big(\frac{n}{4}\Big)=\frac{3n}{4}(\log n - 2)\le c\cdot n\log n
\]
với một $c<1$ khi $n$ đủ lớn (ví dụ $c$ chọn giữa $3/4$ và $1$). Vậy \textbf{trường hợp 3} áp dụng.

\textbf{Kết luận:} $T(n)=\Theta(n\log n)$.

\medskip
\item $T(n)=2T(n/3)+n$.

\textbf{Phân tích:} $a=2,\; b=3 \Rightarrow \log_b a=\log_3 2\approx 0.6309$. $f(n)=n$ có mũ 1 > 0.6309, nên lớn hơn theo đa thức. Kiểm tra regularity:
\[
a\,f(n/b)=2\cdot \frac{n}{3}=\frac{2n}{3}\le c\,n
\]
với $c=\tfrac{2}{3}<1$. Thỏa \textbf{trường hợp 3}.

\textbf{Kết luận:} $T(n)=\Theta(n)$.

\medskip
\item $T(n)=T(n/2)+n$.

\textbf{Phân tích:} $a=1,\; b=2 \Rightarrow \log_b a=\log_2 1=0$. $f(n)=n$ có mũ 1 > 0, nên thuộc \textbf{trường hợp 3}. Kiểm tra regularity: $1\cdot f(n/2)=n/2 \le \frac{1}{2} n$ với $\frac{1}{2}<1$.

\textbf{Kết luận:} $T(n)=\Theta(n)$.

\medskip
\item $T(n)=3T(n/2)+n$.

\textbf{Phân tích:} $a=3,\; b=2 \Rightarrow \log_b a=\log_2 3\approx 1.585$. $f(n)=n = n^{1}$ nhỏ hơn $n^{\log_2 3}$ theo đa thức. Do đó thuộc \textbf{trường hợp 1}.

\textbf{Kết luận:} $T(n)=\Theta\big(n^{\log_2 3}\big)$.

\medskip
\item $T(n)=2T(n/2)+n$.

\textbf{Phân tích:} $a=2,\; b=2 \Rightarrow \log_b a=\log_2 2=1$. $f(n)=n = \Theta(n^{1})$, bằng chính xác $n^{\log_b a}$, do đó là \textbf{trường hợp 2}.

\textbf{Kết luận:} $T(n)=\Theta(n\log n)$.
\end{enumerate}

\section*{Bài 2.1}

\begin{enumerate}[label=\arabic*., leftmargin=1.5em]
    \item Mô tả các bước:\\
    Bước 1: Divide\\
        \hspace*{1.5em}$\bullet$ Nếu mảng đang xét chỉ có 1 phần tử ($left == right$)\\
            \hspace*{3em}-- Nếu phần tử đó = X thì return 1.\\
            \hspace*{3em}-- Ngược lại thì return 0.\\
        \hspace*{1.5em}$\bullet$ Nếu mảng có nhiều hơn 1 phần tử:\\
            \hspace*{3em}-- Tính $mid = (left + right)/2$.\\
            \hspace*{3em}-- Chia mảng thành 2 nửa:\\
                \hspace*{4.5em}$\ast$ Bên trái: $a[left...mid]$\\
                \hspace*{4.5em}$\ast$ Bên phải: $a[mid+1...right]$\\
    Bước 2: Conquer\\
        \hspace*{1.5em}$\bullet$ Gọi đệ quy hàm tìm kiếm trên 2 nửa mảng:\\
            \hspace*{3em}-- $CountX(a, left, mid, X)$\\
            \hspace*{3em}-- $CountX(a, mid+1, right, X)$\\
    Bước 3: Combine\\
        \hspace*{1.5em}$\bullet$ Kết quả trả về: \\
            \hspace*{3em}-- $return CountX(a, left, mid, X) + CountX(a, mid+1, right, X)$
    \item Phân tích độ phức tạp của thuật toán:\\
        \hspace*{1.5em}-- \parbox[t]{0.9\textwidth}{Phân tích: Mỗi lần thuật toán chia mảng thành 2 nửa, gọi đệ quy trên mỗi nửa (kích thước là $n/2$) và cộng kết quả lại.} 
        \hspace*{1.5em}-- Nên ta có phương trình hồi quy : $T(n)=2T(n/2)+O(1)$\\ \hspace*{1.5em}-- \parbox[t]{0.9\textwidth}{Áp dụng Master Theorem với $a=2, b=2, f(n)=O(1)$ ta có $\log_b a = \log_2 2 = 1$. Do đó $f(n) = O(n^{\log_b a - \varepsilon})$ với $\varepsilon = 1$.\\ $\Rightarrow$ Thuộc trường hợp 1 của Master Theorem.}
        \hspace*{1.5em}-- \parbox[t]{0.9\textwidth}{Kết luận: $T(n) = \Theta(n^{\log_2 2}) = \Theta(n)$.}


\end{enumerate}



\end{document}
